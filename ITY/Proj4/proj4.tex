\documentclass[11pt, a4paper]{article}

\usepackage{times}
\usepackage[T1]{fontenc}
\usepackage[left=2cm, text={17cm, 24cm}, top=3cm]{geometry}
\usepackage[czech]{babel}
\usepackage[utf8]{inputenc}
\bibliographystyle{czplain}
\begin{document}
\begin{titlepage}
\begin{center}

\Huge
\textsc{Fakulta informačních technologií Vysoké učení technické v~Brně\\}
\vspace{\stretch{0.382}}
\LARGE
Typografie a publikování -- 4. projekt \\
\Huge{Bibliografické citace}
\vspace{\stretch{0.618}}
\end{center}
{\large 13. dubna 2020 \hfill Jan Hranický -- xhrani02}
\end{titlepage}
\section{\LaTeX}
\subsection{\TeX  vs. \LaTeX}
\TeX je typografický program patřící do rodiny značkovacích jazyků viz \cite{KopkaHelmut2004AGtL}. V~70. letech 20. století \TeX vytvořil americký profesor informatiky Donald E. Knuth. \LaTeX  je sada nástrojů (tzv. maker) pomocí kterých lze uživatelsky přívětivěji používat jazyk \TeX, podrobněji v \cite{KopkaHelmut2004L}. Na vývoji \LaTeX u se dnes převážně podílí tým \LaTeX 3, viz \cite{wikiLaTeX}
\subsection{Princip fungování \LaTeX u}
Vstupem \LaTeX u je textový dokument s~příponou *.tex obsahující příkazy pro sazbu dokumentu. Takovýto soubor je nutno přeložit pomocí překladače. Pro překlad dokumentu lze využít přímo \LaTeX, výstupem potom bude soubor s~příponou *.dvi (Device independent). Takový dokument se poté typicky dále zpracovává do formátu *.pdf. Existují také překladače, které vysázejí přímo dokument ve formátu *.pdf, např. pdftex nebo pdflatex. Detailněji o~principu překladu *.tex souborů se lze dočíst na \cite{BOJKO2011thesis}
\subsection{Kostra Dokumentu}
Každý dokument psaný v~{\LaTeX}u obsahuje:
\begin{itemize}
  \item Hlavičku
  \item Tělo dokumentu
\end{itemize}
Třída dokumentu bývá uvedena na samém začátku dokumentu \cite{onlineRoot}. Prostor mezi tělem a třídou dokumentu se nazývá preambule viz \cite{KopkaHelmut2004L} \cite{Cerny2011}
\subsection{Použití \LaTeX u}
\LaTeX má širokou škálu použití od sazby akademických textů, přes sazbu vektorových obrázků \cite{SedaPavelVGIL} až po reprodukci prastarých textů \cite{ancientText}. Další silnou stránkou \LaTeX u je sazba rovnic. V~čem ale \LaTeX pokulhává, je sazba krátkých plnotextových dokumentů, u~kterých může být vhodnější použít např. MS Word. \cite{abstract}. Více výhod \LaTeX u lze najít v článku \cite{programujteClanek}

\newpage
\bibliography{proj4}
\end{document}
