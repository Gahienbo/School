\documentclass[11pt,a4paper,twocolumn]{article}
\usepackage[czech]{babel}
\usepackage[utf8]{inputenc}
\usepackage[IL2]{fontenc}
\usepackage{times}
\usepackage[left=1.5cm, top=2.5cm, text={18cm, 25cm}]{geometry}
\usepackage{amsmath}
\usepackage{amsthm}
\usepackage{amssymb}

\usepackage[czech]{babel}
\usepackage[utf8]{inputenc}


\newtheorem{definice}{Definice}
\newtheorem{sentence}{Věta}
  {\fontfamily{times}\selectfont}


\begin{document}
\begin{titlepage}
\begin{center}

\Huge
\textsc{Fakulta informačních technologií Vysoké učení technické v~Brně\\}
\vspace{\stretch{0.382}}
\LARGE
Typografie a publikování -- 2. projekt \\
Sazba dokumentů a matematických výrazů
\vspace{\stretch{0.618}}
\end{center}
{\large 2020 \hfill Jan Hranický (xhrani02)}
\end{titlepage}
\section*{Úvod}
V~této úloze si vyzkoušíme sazbu titulní strany, matematických vzorců, prostředí a dalších textových struktur obvyklých pro technicky zaměřené texty (například rovnice (\ref{rov2}) nebo Definice \ref{TS} na straně \pageref{TS}). Pro vytvoření těchto odkazů používáme příkazy \verb|\label|,\verb|\ref| a \verb|\pageref|.

Na titulní straně je využito sázení nadpisu podle op-tického středu s~využitím zlatého řezu. Tento postup bylprobírán na přednášce. Dále je použito odřádkování sezadanou relativní velikostí 0.4em a 0.3em.
\section{Matematický text}
Nejprve se podíváme na sázení matematických symbolů a výrazů v~plynulém textu včetně sazby definic a vět s~využitím balíku \texttt{amsthm}. Rovněž použijeme poznámku podčarou s~použitím příkazu \verb|\footnote|. Někdy je vhodné použít konstrukci \verb|${}$| nebo \verb|\mbox{}| která říká, že (matematický) text nemá být zalomen. V~následující de-finici je nastavena mezera mezi jednotlivými položkami \verb|\item| na 0.05em.
\begin{definice}
\label{TS}
\emph{Turingův stroj(TS)} je definován jako \emph{šestice} tvaru $ M = (Q,\Sigma,\Gamma,\delta,q_0,q_F)$, kde:
\begin{itemize}
  \setlength\itemsep{0.05em}
  \item [$\bullet$] $Q$ je konečná množina vnitřních (řídicích) stavů,\\
  \item [$\bullet$] $\Sigma$ je konečná množina symbolů nazývaná vstupní abeceda, $\Delta\ {\not \in}\ \Sigma$,\\
  \item [$\bullet$] $\Gamma$ je konečná množina symbolů, $\Sigma \subset \Gamma, \Delta \in \Gamma$, nazývaná pásková abeceda,\\
  \item [$\bullet$] $\delta : (Q \setminus \{q_F\})\times\Gamma\rightarrow Q\times(\Gamma\cup\{L,R\})$, kde $L,R\ {\not \in}\   \Gamma$ je parciální přechodová funkce, $a$\\
  \item [$\bullet$] $q_0 \in Q$ je počáteční stav a $q_f \in Q$ je koncový stav.\\
  \end{itemize}
  \end{definice}

  Symbol $\Delta$ značí tzv. \textit{blank} (prázdný symbol), který se vyskytuje na místech pásky, která nebyla ještě použita.

  \textit{Konfigurace} pásky se skládá z~nekonečného řetězce, který reprezentuje obsah pásky a pozice hlavy na tomto \textit{řetězci}. Jedná se o~prvek množiny ${\{\gamma\Delta^\omega|\gamma\in\Gamma^*\} \times \mathbb{N}}$\footnote{Pro libovolnou abecedu $\Sigma$ je $\Sigma^\omega$ množina všech nekonečných řetězců nad $\Sigma$, tj. nekonečných posloupností symbolů ze $\Sigma$.}. \textit{Konfiguraci} pásky obvykle zapisujeme jako $\Delta xyz\underline{z}x\Delta\dots$ (podtržení značí pozici hlavy). \textit{Konfigurace stroje} je pak dána stavem řízení a konfigurací pásky. Formálně se jedná o~prvek množiny $Q \times \{\gamma\Delta^\omega|\gamma\in\Gamma^*\} \times \mathbb{N}$
  \subsection{Podsekce obsahující větu a odkaz}
  \begin{definice}
  {\normalfont Řetězec $w$ nad abecedou $\Sigma$ je přijat jako TS $M$} jestliže $M$ při aktivaci z~počáteční konfigurace pásky $\underline{\Delta}w\Delta\dots$ a počátečního stavu $q_0$ zastaví přechodem do koncového stavu $q_F$, tj. $(q_0,\Delta w\Delta^w, 0){\vdash}{}_M^* (q_F,\gamma,n)$ pro nějáké $\gamma \in \Gamma^*$ a $n \in \mathbb{N}$

  Množinu $L(M) = \{ w | w$ je přijat TS $M \} \subset \Sigma^*$ nazýváme jazyk přijímaný TS $M$.
  \end{definice}

  Nyní si vyzkoušíme sazbu vět a důkazů opět s~použitím balíku \texttt{amsthm}.

  \begin{sentence}
  Třída jazyků, které jsou přijímány TS, odpovídá rekurzivně vyčíslitelným jazykům
  \end{sentence}

  \begin{proof}
  V~důkaze vyjdeme z~Definice 1 a 2.
  \end{proof}


  \section{Rovnice}
  Složitější matematické formulace sázíme mimo plynulý text. Lze umístit několik výrazů na jeden řádek, ale pak je třeba tyto vhodně oddělit, například příkazem \verb|\quad|.
  $$ \sqrt[i]{x^3_i} \quad \text{kde}\ x_i\ \text{je}\ i\text{-té sudé číslo} \quad y^{2 \cdot y_i}_i\neq y^{y^{y_i}_i}_i$$
  V~rovnici (\ref{rov1}) jsou využity tři typy závorek s~různou explicitně definovanou velikostí.
  \begin{eqnarray}
    \label{rov1} x = \bigg\{\Big(\big[a + b\big]*c\Big)^d \oplus 1\bigg\} \\
    \label{rov2} y = \lim_{x\to\infty} \frac{\sin^2x + \cos^2x}{\frac{1}{\log_{10} x}}
  \end{eqnarray}
  V~této větě vidíme, jak vypadá implicitní vysázení limity $\lim_{x\to\infty}f(n)$ v~normálním odstavci textu. Podobně je to i s~dalšími symboly jako $\sum^{n}_{i=1}2^i$ či $\cap_{A \in \beta}A$. V~případě vzorců $\lim\limits_{x\to\infty}f(n)$ a $\sum\limits^{n}_{i=1}2^i$ jsme si vynutili méně úspornou sazbu příkazem \verb|\limits|.

  \section{Matice}
  Pro sázení matic se velmi často používá \texttt{array} a závorky (\verb|\left|, \verb|\right|).
  $$
   \left(
   \begin{array}{ccc}
    a+b & \widehat{\xi + \omega} & \hat{\pi}\\
    \overrightarrow{\textbf{a}} & \overleftrightarrow{AC} & \beta
   \end{array}
   \right) = 1 \iff \mathbb{Q} = \mathcal{R}
  $$
  Prostředí \texttt{array} lze úspěšně využít i jinde.
  $$
    \binom{n}{k} =
    \left\{
    \begin{array}{ll}
    0 & \text{pro } k < 0\ \text{pro } 0\ \leq k \leq n \\
    \frac{n!}{k! (n - k)!} &  \text{ nebo } k > n
    \end{array}
    \right.
  $$

\end{document}
